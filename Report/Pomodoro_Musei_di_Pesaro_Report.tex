\documentclass[hidelinks,12pt,a4paper]{article}
\usepackage[italian]{babel}
\usepackage[utf8]{inputenc}
\usepackage{fourier} 

% Stop hyphenation
\usepackage[none]{hyphenat}

% Justifying text
\emergencystretch 3em

% Remove first empty page
\usepackage{atbegshi}
\AtBeginDocument{\AtBeginShipoutNext{\AtBeginShipoutDiscard}}

% License
\usepackage[
type={CC},
modifier={by-nc-sa},
version={4.0},
]{doclicense}

\begin{document}
	\begin{flushleft}
		
		\title{\textbf{Pomodoro Musei di Pesaro}\\\small{Presentazione progetto}}
		\author{Alice Balestieri\\ Francesco Rombaldoni}
		\date{}
		
		\maketitle
		
		% Adjust page counter
		\setcounter{page}{1}
		\newpage
		\tableofcontents
		\newpage
		
		\section{Le nostre competenze}
			\subsection{Chi siamo}
			"Pomodoro-Musei di Pesaro" è una iniziativa nata dalla nostra amicizia e dal desiderio di contribuire a migliorare in maniera innovativa alcune problematiche/aspetti riscontrati all'interno dell'ambito museale di Pesaro.\\
			Noi crediamo che l'unione tra il mondo artistico e il mondo tecnico informatico, possa essere una combinazione vincente non solo per poter risolvere efficacemente le nuove problematiche legate all'ambiente museale, ma anche, per valorizzare quest'ultimo in un periodo dove il tema della informatizzazione sta diventando sempre più importante.\\
			Il team di lavoro è composto da: Alice, curatrice della parte artistica e informativa, e da Francesco invece curatore della parte informatica e di gestione delle risorse. I nostri "nickname" su internet sono rispettivamente: r0s4dip3sar0 e R0mb0. 
			
			
			\subsection{I nostri studi}
			La nostra formazione formazione parte da ambienti e origini differenti, in particolare Alice
			% Parte di Alice
			e Francesco dopo essersi diplomato al Liceo Scientifico ha subito proseguito gli studi iniziando la triennale di "informatica applicata" presso l'Università di Urbino. Ora quasi a termine degli studi della triennale ha deciso di proseguire la formazione immatricolandosi presso la magistrale dello stesso corso. 
			
			
			\subsection{Le nostre Capacità}
			Le abilità del nostro team risiedono dietro alle conoscenze artistiche di Alice
			% Parte di Alice
			e alle conoscenze tecniche di Francesco, che pur avendo intrapreso un percorso di studi tecnico non ha mai abbandonato le sue passioni, come la passione della musica, della fotografie e della modellazione/stampa in 3d. Le sue principali abilità all'interno del progetto risiedono nella pianificazione e gestione del lavoro ad alto livello, nonché una esperienza dietro le strategie di lavoro aziendali e di costruzione di un "workteam". Dal punto di vista tecnico/informatico, le competenze risiedono nella buona padronanza di strumenti di lavoro asincrono come "GitHub", in una buona conoscenza dei più comuni linguaggi di programmazione, tra cui di particolare rilevanza il linguaggio \LaTeX usato per la scrittura della documentazione del progetto, in una discreta conoscenza di programmi per la manipolazione di video, foto e audio e infine una base di elettronica usata per il montaggio di impianti audio e video fino al livello semi-professionale.
			
		
		\section{Obiettivo del progetto}
			\subsection{Presentazione obiettivo}
			L'obiettivo del progetto è quello di fornire sia agli operatori dei musei che anche ai turisti delle guide rigorose per la spiegazione delle opere contenute nei musei, oltre che fornire una serie di strumenti/documenti avanzati per poter presentare le opere in maniera attraente ed innovativa.
			
			\subsection{Strategia per il raggiungimento}
			La nostra strategia risiede in primo luogo nella scelta di voler lavorare ad alto livello secondo gli standard europei e dove possibile internazionali.\\
			L'obiettivo finale è quello di condurre un progetto di livello 3 secondo lo standard: "Capability Maturity Model (CMM)". Per questo scopo si sono fin da subito adoperati strumenti come "GitHub" per garantire l'intera tracciabilità del processo di sviluppo, ed inoltre, si sono adottate strategie di lavoro ad alto livello per l'organizzazione del team, come la strategia di lavoro a spirale spesso utilizzata per lo sviluppo di progetti simili al nostro. Altro particolare fondamentale per il raggiungimento dell'obiettivo, risiede nel gran numero di documentazione prodotta, non solo per favorire ulteriormente la tracciabilità del lavoro, ma anche per la sua pianificazione e gestione.\\
			Per quanto riguarda il materiale prodotto, si utilizzano strumenti e licenze "open-source", il tutto rispettando i regolamenti e gli standard europei, per questo motivo si è deciso di sviluppare il progetto in un linguaggio di programmazione non proprietario come il \LaTeX, oltre che si è fin da subito usato una piattaforma di sviluppo di progetti pubblici/"open-source" certificata come "GitHub", sempre per fare in modo di rispettare gli standard europei. Riguardo ai contenuti non auto-prodotti, come le immagini delle opere, secondo gli standard si ze riportata nella pagina del progetto le licenze relative a tutti i contenuti utilizzati.
		
		\section{Presentazione del progetto}
			\subsection{Presentazione linguaggio \LaTeX}
			\subsection{Presentazione piattaforma GitHub per lo sviluppo asincrono}
			\subsection{Presentazione impostazione dell'ambiente di lavoro}
			\subsection{Presentazione strategia lavorativa tenuta durante lo sviluppo}
			\subsection{Presentazione dei documenti del progetto}
		
		\section{Note finali}
		
	\end{flushleft}
\end{document}