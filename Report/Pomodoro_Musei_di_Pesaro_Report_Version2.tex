\documentclass[hidelinks,12pt,a4paper]{article}
\usepackage[italian]{babel}
\usepackage[utf8]{inputenc}
\usepackage{fourier} 

% Stop hyphenation
\usepackage[none]{hyphenat}

% Justifying text
\emergencystretch 3em

% Remove first empty page
\usepackage{atbegshi}
\AtBeginDocument{\AtBeginShipoutNext{\AtBeginShipoutDiscard}}

% License
\usepackage[
type={CC},
modifier={by-nc-sa},
version={4.0},
]{doclicense}

\begin{document}
	\begin{flushleft}
		
		\title{\textbf{Pomodoro Musei di Pesaro}\\\small{Presentazione progetto}}
		\author{Alice Balestieri\\ Francesco Rombaldoni}
		\date{}
		
		\maketitle
		
		% Adjust page counter
		\setcounter{page}{1}
		\newpage
		\tableofcontents
		\newpage
			
			\section{Le nostre competenze}
				\subsection{Chi siamo}
				"Pomodoro-Musei di Pesaro" è una iniziativa nata dalla nostra amicizia e dal desiderio di contribuire a migliorare in maniera innovativa alcune problematiche/aspetti riscontrati all'interno dell'ambito museale di Pesaro.\\
				Noi crediamo che l'unione tra il mondo artistico e il mondo tecnico informatico, possa essere una combinazione vincente non solo per poter risolvere efficacemente le nuove problematiche legate all'ambiente museale, ma anche, per valorizzare quest'ultimo in un periodo dove il tema della informatizzazione sta diventando sempre più importante.\\
				Il team di lavoro è composto da: Alice, curatrice della parte artistica e informativa, e da Francesco invece curatore della parte informatica e di gestione delle risorse. I nostri "nickname" su internet sono rispettivamente: r0s4dip3sar0 e R0mb0. 
				
				\subsection{I nostri studi}
				La nostra formazione parte da ambienti e origini differenti, in particolare Alice
				% Parte di Alice
				e Francesco dopo essersi diplomato al Liceo Scientifico ha subito proseguito gli studi iniziando la triennale di "informatica applicata" presso l'Università di Urbino. Ora quasi a termine degli studi della triennale ha deciso di proseguire la formazione immatricolandosi presso la magistrale dello stesso corso. 
				
				\subsection{Le nostre Capacità}
				Le abilità del nostro team risiedono dietro alle conoscenze artistiche di Alice
				% Parte di Alice
				e alle conoscenze tecniche di Francesco, che pur avendo intrapreso un percorso di studi tecnico non ha mai abbandonato le sue passioni, come la passione della musica, della fotografie e della modellazione/stampa in 3d. Le sue principali abilità all'interno del progetto risiedono nella pianificazione e gestione del lavoro ad alto livello, nonché una esperienza dietro le strategie di lavoro aziendali e di costruzione di un "workteam". Dal punto di vista tecnico/informatico, le competenze risiedono nella buona padronanza di strumenti di lavoro asincrono come "GitHub", in una buona conoscenza dei più comuni linguaggi di programmazione, tra cui di particolare rilevanza il linguaggio \LaTeX usato per la scrittura della documentazione del progetto, in una discreta conoscenza di programmi per la manipolazione di video, foto e audio e infine una base di elettronica usata per il montaggio di impianti audio e video fino al livello semi-professionale.
			
			\section{Presentazione del progetto}
				\subsection{Obiettivo del progetto}
				L'obiettivo del progetto è quello di fornire sia agli operatori dei musei che anche ai turisti delle guide rigorose per la spiegazione delle opere contenute nei musei, oltre che fornire una serie di strumenti/documenti avanzati per poter presentare le opere in maniera attraente ed innovativa.
				
				\subsection{Il lavoro svolto}
				Per via della natura del progetto molto legata all'ambito museale, ovvero all'ambito pubblico, si è pensato di condurre fin da subito il lavoro in modalità "open-source".\\
				Il progetto è stato licenziato e condotto, rispettivamente con l'ausilio di licenze e strumenti ritenuti etici a livello internazionale. La licenza che è stata scelta per tutelare il progetto è la "common creator 4.0" la quale è la principale licenza usata in territorio europeo per la pubblicazione di documenti di natura pubblica. In particolare la licenza usata per tutelare il progetto è la "Creative Commons Attribution-NonCommercial-ShareAlike 4.0 International License". In linea con l'etica "open-source" sono state riportate (nei documenti "ReadMe.md" messi in evidenza nella piattaforma che ospita online il progetto) tutte le licenze e i copyright (con riferimento a link di provenienza) dei materiali usati nel progetto che non sono stati auto prodotti.\\
				Gli strumenti usati per lo sviluppo sono stati: la piattaforma "GitHub" per la gestione del progetto e l'accoppiata del linguaggio \LaTeX con il compilatore "Tex studio". Tutti gli strumenti usati sono anch'essi "open-source" come il progetto. In particolare il linguaggio \LaTeX è stato scelto siccome è un linguaggio di programmazione avanzato con il quale si possono redarre documenti formali, del quale la chia
	\end{flushleft}
\end{document}