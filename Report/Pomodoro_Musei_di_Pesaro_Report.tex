\documentclass[hidelinks,12pt,a4paper]{article}
\usepackage[italian]{babel}
\usepackage[utf8]{inputenc}
\usepackage{fourier} 

% Stop hyphenation
\usepackage[none]{hyphenat}

% Justifying text
\emergencystretch 3em

% Remove first empty page
\usepackage{atbegshi}
\AtBeginDocument{\AtBeginShipoutNext{\AtBeginShipoutDiscard}}

% License
\usepackage[
type={CC},
modifier={by-nc-sa},
version={4.0},
]{doclicense}

\begin{document}
	\begin{flushleft}
		
		\title{\textbf{Pomodoro Musei di Pesaro}\\\small{Presentazione progetto}}
		\author{Alice Balestieri\\ Francesco Rombaldoni}
		\date{}
		
		\maketitle
		
		% Adjust page counter
		\setcounter{page}{1}
		\newpage
		\tableofcontents
		\newpage
		
		\section{Le nostre competenze}
			\subsection{Chi siamo}
			"Pomodoro-Musei di Pesaro" è una iniziativa nata dalla nostra amicizia e dal desiderio di contribuire a migliorare in maniera innovativa alcune problematiche/aspetti riscontrati all'interno dell'ambito museale di Pesaro.\\
			Noi crediamo che l'unione tra il mondo artistico e il mondo tecnico informatico, possa essere una combinazione vincente non solo per poter risolvere efficacemente le nuove problematiche legate all'ambiente museale, ma anche, per valorizzare quest'ultimo in un periodo dove il tema della informatizzazione sta diventando sempre più importante.\\
			Il team di lavoro è composto da: Alice, curatrice della parte artistica e informativa, e da Francesco invece curatore della parte informatica e di gestione delle risorse. I nostri "nickname" su internet sono rispettivamente: r0s4dip3sar0 e R0mb0. 
			
			
			\subsection{I nostri studi}
			La nostra formazione formazione parte da ambienti e origini differenti, in particolare Alice
			% Parte di Alice
			e Francesco dopo essersi diplomato al Liceo Scientifico ha subito proseguito gli studi iniziando la triennale di "informatica applicata" presso l'Università di Urbino. Ora quasi a termine degli studi della triennale ha deciso di proseguire la formazione immatricolandosi presso la magistrale dello stesso corso. 
			
			
			\subsection{Le nostre Capacità}
			Le abilità del nostro team risiedono dietro alle conoscenze artistiche di Alice
			% Parte di Alice
			e alle conoscenze tecniche di Francesco, che pur avendo intrapreso un percorso di studi tecnico non ha mai abbandonato le sue passioni, come la passione della musica, della fotografie e della modellazione/stampa in 3d. Le sue principali abilità all'interno del progetto risiedono nella pianificazione e gestione del lavoro ad alto livello, nonché una esperienza dietro le strategie di lavoro aziendali e di costruzione di un "workteam". Dal punto di vista tecnico/informatico, le competenze risiedono nella buona padronanza di strumenti di lavoro asincrono come "GitHub", in una buona conoscenza dei più comuni linguaggi di programmazione, tra cui di particolare rilevanza il linguaggio \LaTeX usato per la scrittura della documentazione del progetto, in una discreta conoscenza di programmi per la manipolazione di video, foto e audio e infine una base di elettronica usata per il montaggio di impianti audio e video fino al livello semi-professionale.
			
		
		\section{Obiettivo del progetto}
			\subsection{Presentazione obiettivo}
			L'obiettivo del progetto è quello di fornire sia agli operatori dei musei che anche ai turisti delle guide rigorose per la spiegazione delle opere contenute nei musei, oltre che fornire una serie di strumenti/documenti avanzati per poter presentare le opere in maniera attraente ed innovativa.
			
			\subsection{Strategia per il raggiungimento}
			La nostra strategia risiede in primo luogo nella scelta di voler lavorare ad alto livello secondo gli standard europei e dove possibile internazionali.\\
			L'obiettivo finale è quello di condurre un progetto di livello 3 secondo lo standard: "Capability Maturity Model (CMM)". Per questo scopo si sono fin da subito adoperati strumenti come "GitHub" per garantire l'intera tracciabilità del processo di sviluppo, ed inoltre, si sono adottate strategie di lavoro ad alto livello per l'organizzazione del team, come la strategia di lavoro a spirale spesso utilizzata per lo sviluppo di progetti simili al nostro. Altro particolare fondamentale per il raggiungimento dell'obiettivo, risiede nel gran numero di documentazione prodotta, non solo per favorire ulteriormente la tracciabilità del lavoro, ma anche per la sua pianificazione e gestione.\\
			Per quanto riguarda il materiale prodotto, si utilizzano strumenti e licenze "open-source", il tutto rispettando i regolamenti e gli standard europei, per questo motivo si è deciso di sviluppare il progetto in un linguaggio di programmazione non proprietario come il \LaTeX, oltre che si è fin da subito usato una piattaforma di sviluppo di progetti pubblici/"open-source" certificata come "GitHub", sempre per fare in modo di rispettare gli standard europei. Riguardo ai contenuti non auto-prodotti, come le immagini delle opere, secondo gli standard si ze riportata nella pagina del progetto le licenze relative a tutti i contenuti utilizzati.
		
		\section{Presentazione del progetto}
			\subsection{Presentazione linguaggio \LaTeX}
			Il \LaTeX è un linguaggio di "markup" creato da Leslie Lamport nel 1985, il suo scopo principale è quello di produrre documenti dall'alta qualità tipografica.\\
			Al contrario di editor (o word processor) più conosciuti (come ad esempio Microsoft Word) che si basano sul paradigma WYSIWYG (What You See Is What You Get, cioè ciò che vedi è quello che ottieni), con il \LaTeX ci si concentra sul contenuto dei testi, mantenendo un alto grado di libertà per quanto riguarda l'impaginazione.\\
			Questo approccio viene anche definito WYSIWYM (What You See Is What You Mean, cioè ciò che vedi è quello che intendi): con LaTeX l'autore\"designer" inizialmente può occuparsi delle convenzioni da usare, ma una volta fissate queste, l'attenzione rimane soltanto sul contenuto del testo. Grazie anche al fatto che i linguaggi di "markup" come gli altri linguaggi di programmazione permettono di automatizzare molte operazioni, come ad esempio la generazione degli indici cliccabili e l'inserimento "fine" delle immagini. \\
			Il \LaTeX oltre a documenti "classici" può inoltre produrre presentazioni della stessa resa grafica grazie alla classe Beamer.
			Per via della sua natura assai duttile, il \LaTeX è utilizzabile anche al di fuori degli ambiti "scientifici"  nei quali ne viene fatto largo uso, ma può prestarsi anche all'utilizzo in ambito umanistico.\\
			Un'altra caratteristica importante del \LaTeX è che questo linguaggio è stato rilasciato sotto licenza "open-source", grazie a ciò , è diventato ben presto uno standard in tutti gli ambienti dove viene richiesta una maggiore formalità, in quanto è un linguaggio molto sicuro che rispetta gli standard europei ed internazionali per la scrittura di documenti.
			
			\subsection{Presentazione piattaforma GitHub per lo sviluppo asincrono}
			GitHub è la principale implementazione online di Git ovvero il software più diffuso per il controllo di versione distribuito.\\
			Le principali caratteristiche che contraddistinguono questa piattaforma di sviluppo asincrono rispetto al altre sono: la facilità con la quale è possibile creare un progetto online (di natura sia "open" che "closed"-source), la possibilità tramite file in "markdown" di poter descrivere in maniera semplice ed efficace il progetto e parti di esso, permette di tenere traccia di tutte le interazione avvenute nel progetto con i relativi commenti garantendo oltretutto la prova che il progetto possiede una determinata paternità, siccome sia la la piattaforma che il software ivi contenuto è opensource questo sistema è diventato standard per lo sviluppo e mantenimento di progetti pubblici, altra sua grande caratteristica è il fatto di offrire delle grandi possibilità di automatizzare il lavoro permettendo quindi di concentrarsi sul lavoro stesso e non sulla gestione ,essendo un servizio su internet permette inoltre di tenere al sicuro i dati del progetto, siccome l'ambiente di lavoro condiviso non è locale alle macchine, ma è online.\\
			
			\subsection{Presentazione impostazione dell'ambiente di lavoro}
			Fin dall'inizio si è deciso di sviluppare il progetto in modalità "open-source", in primo luogo per rispettare le leggi che impongono che i lavori che se basano su cose pubbliche, devono poi essere pubblici a loro volta, ma anche per poter accogliere le correzione proposte da tutti coloro che hanno piacere di contribuire liberamente al progetto.\\
			Sulla base dello sviluppo "open-source" è stato deciso di usare la piattaforma "GitHub", sia per il fatto che è attualmente una piattaforma certificata per lo sviluppo di lavori pubblici, ma anche per il fatto che è una piattaforma per la collaborazione asincrona che facilità il lavoro ad alto livello.\\
			Infatti il progetto è stato sempre condotto con delle strategie di lavoro ad alto livello, in particolare grazie alla piattaforma "GitHub" si è tenuta traccia di tutto il processo lavorativo a partire dalla creazione dei primi documenti. Oltre a ciò, è stata creta anche la documentazione che tiene traccia di tutte le riunioni fatte per la gestione del progetto.\\
			Per rimanere allineati alla filosofia "open-source" sono stati scelti e utilizzati strumenti a licenze appartenenti alla stessa filosofia, in particolare l'intero materiale e progetto, è licenziato sotto licenza: "Common-Creator", la quale è una licenza per la tutela della paternità (largamente usata per la difesa dei documenti pubblicati su internet) gestita da una azienda "no-profit"  internazionale, per questo motivo, la licenza "Common-Creator" e una delle licenze individuate dall'unione europea come standard di sviluppo di progetti pubblici. Inoltre nello spirito di progetti simili, sono state riportate tutte le licenze e i riferimenti dei materiali (come le foto) sfruttati per il progetto.\\
			Gli strumenti usati (tutti "open-source") sono: il linguaggio \LaTeX usato in accoppiata con "l'editor TeX-Studio", per lo sviluppo e compilazione dei documenti, mentre per interagire con il servizio online di "GitHub" è stata usata l'applicazione "GitHub-Desktop".
			\bigskip
			L'ambiente di lavoro è stato impostato con delle strategie di sviluppo prescrittivo, in particolare  
			
			\subsection{Presentazione strategia lavorativa tenuta durante lo sviluppo}
			\subsection{Presentazione dei documenti del progetto}
		
		\section{Note finali}
			\subsection{Lavori aggiuntivi per arricchire l'offerta}
		
	\end{flushleft}
\end{document}